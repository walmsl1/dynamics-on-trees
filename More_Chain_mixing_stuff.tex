\documentclass{article}

\usepackage{amssymb}
\usepackage{amsmath}
\usepackage{amsfonts}
\usepackage{amsthm}
\usepackage{textcomp}
\usepackage{mdwlist}
\usepackage{enumerate}
\usepackage{graphicx}
\usepackage{multicol,tikz}

\newcommand{\babble}[1]{\marginpar{\flushleft\scriptsize\textsf{#1}}}
\newcommand{\insertpic}[2][0.4]{\begin{center}\includegraphics*[scale=#1]{Figures/#2}\end{center}}
\newcommand{\uu}{\mathbf{u}}
\newcommand{\vv}{\mathbf{v}}
\newcommand{\ww}{\mathbf{w}}
\newcommand{\xx}{\mathbf{x}}

\def\R{\mathbb{R}}
\def\Nul{\operatorname{Nul}}
\def\Col{\operatorname{Col}}
\def\Row{\operatorname{Row}}
\def\tr{\operatorname{tr}}

\oddsidemargin=0in
\textwidth=5.75in
\topmargin=-0.5in
\textheight=9in
\renewcommand{\marginparwidth}{80pt}
\newcommand{\RR}{\mathbb{R}}

\begin{document}
\begin{center}
{\bf 
Chain Transitivity Proof for Directed Trees\\

%\centerline{\hspace{1.75cm}\includegraphics[width=0.42\textwidth]{E.jpg}}%
}
\end{center}
\medbreak\medbreak


\begin{enumerate}

\item Lemma $0.2$

Let $\displaystyle\sum^{\infty}_{n-1}
\hspace{0.2cm}\sum_{u\in Chi^n(v)}\hspace{-0.3cm}\lambda(v \rightarrow u) = +\infty$, then for any $b\in \mathbb{C}$ and $v\in (V,E)$, there exists an $\epsilon$-chain from $0$ to $be_v$, $z^{(0)}=0,z^{(1)},\dots,z^{(m)}=be_v$.

\begin{proof}
    There exists an $m$ large enough such that $t:=\displaystyle\sum^{m-1}_{n=1} \sum_{u\in Chi^n(v)}\hspace{-0.4cm} \lambda(v\rightarrow u) + 1 > \frac{|b|}{\delta}$.

    For $j=\{1,\dots,m\}$, define $r^{(j)}:=\frac{b}{t}\hspace{-0.4cm}\displaystyle\sum_{u\in Chi^{m-j}(v)}\hspace{-0.4cm}e_u$.
    
    Let $k_j=|chi^j(v)|$ and $k=max\{k_j\}$.

    Let $\delta>0$ and $\epsilon=k\delta$ ($\epsilon$ is still arbitrary and greater than $0$)


    Define $z^{(j)}:= B_\lambda z^{(j-1)} +r^{(j)}$.

    So, $||z^{(j)}-B_\lambda z^{(j-1)}||=||r^{(j)}||=\frac{k_j|b|}{t} \leq \frac{k|b|}{t}\leq k\delta=\epsilon$.

    Then, $z^{(m)}=B_\lambda^m z^{(0)}+B_\lambda^{m-1}+r^{(1)}+\dots+B_\lambda r^{m-1}+r^{(m)}$.

    For $j+\{1,\dots,m-1\}$, $B_\lambda^{m-j}r^{(j)}=\frac{b}{t}B_\lambda^{m-j}\hspace{-0.4cm}\displaystyle\sum_{u\in Chi^{m-j}(v)}\hspace{-0.4cm}e_u = \frac{b}{t}(\hspace{-0.2cm}\displaystyle\sum_{u\in Chi^{m-j}(v)} \hspace{-0.4cm}\lambda(v\rightarrow u))$.

    So, $z^{(m)}=0+\frac{b}{t}(\displaystyle\sum^{m-1}_{n=1}\sum_{u \in Chi^n(v)}\hspace{-0.4cm}\lambda(v\rightarrow u))e_v+\frac{b}{t}e_v$

    $=\frac{b}{t}(1+\displaystyle\sum^{m-1}_{n=1}\sum_{u \in Chi^n(v)}\lambda(v\rightarrow u))e_v=\frac{b}{t}\cdot t e_v=be_v$.
\end{proof}


\item Proposition: Let $(V,E)$ be a directed tree. The, the backwards shift $B_\lambda$ is chain transitive/mixing/recurrent if and only if:

\medskip

\centerline{$\forall v\in V$, $\displaystyle\sum^{\infty}_{n-1}
\hspace{0.2cm}\sum_{u\in Chi^n(v)}\hspace{-0.3cm}\lambda(v \rightarrow u) = +\infty$.}

\begin{proof}

($\Leftarrow$) We will show that there exists a $1$-chain from $x$ to $y$.

There exists an $l$ big enough so $\tilde{x}=\displaystyle\sum^l_{n=0} \sum_{u\in Chi^n(root)}\hspace{-0.4cm} x_u e_u$. 
(Where $x_n$ is the value at $u$ in vector $x$)
such that $||B_\lambda x-\tilde{x}||<1$. This $l$ also ensures that $\tilde{y}=\dots$ such that $||y-B_\lambda\tilde{y}||<1$.

We obtains a $1-chain$ of the form:

\medskip

\centerline{$x,\tilde{x},B_\lambda\tilde{x},B_\lambda^2\tilde{x},\dots,B_\lambda^{l-1}\tilde{x},0,z_1,z_2,\dots,z_m,y$}

\smallskip

Then let $\zeta$ be the number of points in $\tilde{y}$. Thus, $\zeta= \displaystyle\sum^l_{n=1}\sum_{u\in Chi^n(v)}\hspace{-0.35cm}1$.



By \textit{Lemma 0.2}, there exists a $\frac{1}{\zeta}$-chain from $0$ to $b_ve_v$ $\forall v\in\tilde{y}$.

Since we can insert $0$'s into the beginning of each chain, we can assume that all chains have length $m$. That is, we can assure there exists a $\frac{1}{\zeta}$-chain of the form:

\centerline{$x^{(i)}_0=x^{(i)}_1,x^{(i)}_2,\dots,x^{(i)}_m=b_ve_v$.}

An easy application of the triangle inequality shows that the chain $z_0=0,z_1,z_2,\dots,z_m$ defined by $z_i := \displaystyle\sum^m_{j=1} x^{(j)}_i$ for $i\in \{1,\dots,l\}$, is a $1$-chain from $0$ to $b_1e_1+\dots+b_le_l=\tilde{y}$, as desired. 


($\Rightarrow$) We want to show that is the backwards shift is chain transitive, then the sum of the weight sums of all the vertices in the directed tree diverges to $+\infty$. Fix $t>1$, and let $v\in V$. Assume chain transitivity. Thus, there exists some $1$-chain, 

\smallskip

\centerline{$te_v = x_0, x_1,\dots,x_n = te_v$}

\smallskip

such that $||B_\lambda x_i-x_{i+1}||<1$ $\forall i \in (0,\dots,n-1)$.

Therefore, $x_{i+1}=B_\lambda x_i +r^{(1)}$ with $||r^{(i)}||<1$.

Then we have,

\smallskip

\centerline{$te_v=x_n=B_\lambda^n x_0 + B^{n-1}_\lambda r^{0}+B^{n-2}_\lambda r^{1}+\dots+B_\lambda r^{n-2}+r^{(n-1)}$.}

Then, no matter how many children $v$ has, $(B^n_\lambda x_0)(v)=0$.

Thus,

\centerline{$(B^n_\lambda r^{(i)})(v)= \hspace{-0.45cm} \displaystyle\sum_{u\in {Chi}^n(v)}\hspace{-0.4cm}\lambda(v\rightarrow u) r^{(i)}_u$.}

Then we write,

\medskip

\centerline{$t=0+\hspace{-0.6cm} \displaystyle\sum_{u\in {Chi}^{n-1}(v)}\hspace{-0.6cm}\lambda(v\rightarrow u)r_u^{(0)}
+
\hspace{-0.55cm} \displaystyle\sum_{u\in {Chi}^{n-2}(v)}\hspace{-0.55cm}\lambda(v\rightarrow u)r_u^{(1)}
+\dots+
\hspace{-0.4cm} \displaystyle\sum_{u\in {Chi}(v)}\hspace{-0.35cm}\lambda(v\rightarrow u)r_u^{(n-2)}
+
r_u^{(n-1)}$.}

Since $r^{(i)}<1$ then we can substitute $1$ in for the $r$-terms, and the result must be larger. Thus, 

\centerline{$t\leq \hspace{-0.6cm} \displaystyle\sum_{u\in {Chi}^{n-1}(v)}\hspace{-0.6cm}\lambda(v\rightarrow u)
+
\hspace{-0.55cm} \displaystyle\sum_{u\in {Chi}^{n-2}(v)}\hspace{-0.55cm}\lambda(v\rightarrow u)
+\dots+
\hspace{-0.4cm} \displaystyle\sum_{u\in {Chi}(v)}\hspace{-0.35cm}\lambda(v\rightarrow u)
+
1$}

\medskip

\centerline{$ = \displaystyle\sum^{n-1}_{m-1}
\hspace{0.2cm}\sum_{u\in Chi^m(v)}\hspace{-0.3cm}\lambda(v \rightarrow u) +1 \geq t$.}

Then since $t$ is arbitrarily large, we have,

\medskip

\centerline{$ \displaystyle\sum^{\infty}_{m-1}
\hspace{0.2cm}\sum_{u\in Chi^m(v)}\hspace{-0.3cm}\lambda(v\rightarrow u) = +\infty$} 
as desired.



\end{proof}



%%%%%%%%%%%%%%%%%%%%%%%%%%%%%%%%%%%%%%%%


\item Lemma $0.2$

Let $\displaystyle\sum^{\infty}_{n=1} \sum_{u \in Chi^n(v)} \lambda(v \to u) = +\infty$. Then for any $b \in \mathbb{C}$ and $v \in (V,E)$, there exists an $\epsilon$-chain from $0$ to $be_v$, $z^{(0)}=0, z^{(1)}, \dots, z^{(m)}=be_v$.

\begin{proof}
    There exists an $m$ large enough such that $t:=\displaystyle\sum^{m-1}_{n=1} \sum_{u \in Chi^n(v)} \lambda(v \to u) + 1 > \frac{|b|}{\delta}$.

    Define $r^{(j)}:=\frac{b}{t} \sum_{u \in Chi^{m-j}(v)} e_u$.
    
    Let $\delta>0$ and $\epsilon = \delta$.
    
    Define $z^{(j)}:= B_\lambda z^{(j-1)} + r^{(j)}$.

    Then, $||z^{(j)}-B_\lambda z^{(j-1)}|| = ||r^{(j)}|| = \frac{|b|}{t} \leq \delta = \epsilon$.

    So, $z^{(m)} = \frac{b}{t} (1 + \sum^{m-1}_{n=1} \sum_{u \in Chi^n(v)} \lambda(v \to u)) e_v = be_v$.
\end{proof}

$\tilde{\tilde{x}}$ 
$\overset{\infty}{x}_0$
$\varepsilon \epsilon$


%%%%%%%%%%%%%%%%%%%%%%%%%%%%%%%%%%%%%%%%



Lemma: $\forall v \in (V,E)$,
$\displaystyle\sum^\infty_{n=1}\{sup|\lambda(u\rightarrow v)|:u\in Chi^n(v)\}=+\infty$ $\Rightarrow$ $\epsilon$-chain from $0$ to $be_v$.

\begin{proof}
    Let $\displaystyle\sum^\infty_{n=1}\{sup|\lambda(v\rightarrow u)|:u\in Chi^n(v)\}=+\infty$ for all $v\in (V,E)$. 
    
    \medskip 
    
    Let $b\in\mathbb{K}$, let $\epsilon>0$, and let $v\in(V,E)$. 

    There exists an $m$ large enough such that $\displaystyle\sum^{m-1}_{n=1}\{sup|\lambda(v\rightarrow u)|:u\in Chi^n(v)\}+1>\frac{|b|}{\epsilon}$.

    Then there also exists $u_i\in Chi^i(v)$, $(i\in\{1,\dots,m-1\})$ such that:

    \medskip
    
    \centerline{$t:=\displaystyle\sum^{m-1}_{n=1}\lambda(v\rightarrow u_n)+1>\frac{|b|}{\epsilon}$,} 
    
    because we can find $\lambda(v\rightarrow u_n)$ as close to $sup|\lambda(v\rightarrow u)|:u \in Chi^n(v)$ as we want by the definition of a supremum. 

    For $j\in \{1.\dots,m\}$, define $r^{(j)}:=\frac{b}{t}e_{u_{m-j}}$.

    Then define $z^{(j)}:=B_\lambda z^{(j-1)}+r^{(j)}$, so $||z^{(j)}-B_\lambda z^{(j-1)}||=||r^{(j)}||=\frac{b}{t}<\epsilon$.

    Now we have an $\epsilon$-chain: $z^{(0)}=0,z^{(1)},\dots,r^{(m)}=be_v$.

    So
    \begin{enumerate}
        \item $z^{(m)}=B_\lambda^m z^{(0)}+B_\lambda^{m-1}r^{(1)}+\dots+B_\lambda r^{(m-1)}+r^{(m)}$.

        \item $B_\lambda^{m-j}r^{(j)}= B_\lambda^{m-j}\frac{b}{t}e_{u_{m-j}}=\frac{b}{t}\lambda(v \rightarrow u_{m-j})e_v$.
    \end{enumerate}
   

    Thus, $z^{(m)}= \frac{b}{t}\lambda(v\rightarrow u_{m-1})e_v+\dots+\frac{b}{t}\lambda(v\rightarrow u_1)e_v=\frac{b}{t}(\displaystyle\sum^{m-1}_{n=1}\lambda(v\rightarrow u_n)+1)e_v=\frac{b}{t}\cdot te_v=be_v$, as desired.
\end{proof}

\textbf{Theorem:} $\displaystyle\sum sup|\lambda()|=+\infty$, $\Rightarrow$ chain mixing.

$1$-chain: $x,\tilde{x},B_\lambda\tilde{x}, B^2_\lambda\tilde{x},\dots,0,z_1,\dots,z_m=\tilde{\tilde{y}},\tilde{y},y$...




%%%%%%%%%%%%%%%%%%%%%%%%%%%%%%%%%%%%%%%%

\end{enumerate}
\end{document}